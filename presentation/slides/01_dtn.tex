\section{\acf{DTN}}

\begin{frame}
  \frametitle{\acf{DTN}}

  \begin{columns}
  \column{0.6\textwidth}
  \begin{itemize}
  \item Situations without a reliable uplink
    \begin{itemize}
    \item Environmental monitoring in remote areas
    \item Destroyed telecommunication infrastructure
    \item Internet access is blocked
    \end{itemize}

  \onslide<3->{
  \item In \acs{DTN}, data is transmitted in a store-carry-forward fashion
    \begin{itemize}
    \item Hop-by-hop transport
    \item Opportunistic or scheduled contacts to neighbors
    \item Allows large time window between two transmissions
    \end{itemize}
  }
  \end{itemize}

  \column{0.4\textwidth}
  \end{columns}

  \begin{textblock*}{0.4\textwidth}(0.7\textwidth,1.75cm)
    \includegraphics<2>[width=\linewidth,height=\textheight,keepaspectratio]{include/dtn-example-1}
    \includegraphics<3>[width=\linewidth,height=\textheight,keepaspectratio]{include/dtn-example-2}
    \includegraphics<4>[width=\linewidth,height=\textheight,keepaspectratio]{include/dtn-example-3}
  \end{textblock*}
\end{frame}

\begin{frame}
  \frametitle{DTN7}

  This brings us to DTN7\dots

  \begin{itemize}
  \item Free and open-source \acs{DTN} software
  \item Written in the Go programming language
  \item Modularized design, easy to extend
  \item Implementation of the recently released \acf{BP}
  \end{itemize}
\end{frame}
